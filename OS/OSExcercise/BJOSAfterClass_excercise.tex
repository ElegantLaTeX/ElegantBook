\documentclass[lang=cn,newtx,10pt,scheme=chinese]{../../elegantbook}

\title{课后提升练习题}
\subtitle{北街学长倾力之作}

\author{北街}
% \institute{Elegant\LaTeX{} Program}
\date{2022/12/31}
\version{1.0}
% \bioinfo{自定义}{信息}

% \extrainfo{注意:本模板自 2023 年 1 月 122222 日开始,不再更新和维护!}

\setcounter{tocdepth}{3}

\logo{../../figure/logo-blue.png}
\cover{../../figure/cover.jpg}

% 本文档命令
\usepackage{array}
\newcommand{\ccr}[1]{\makecell{{\color{#1}\rule{1cm}{1cm}}}}

% 修改标题页的橙色带
\definecolor{customcolor}{RGB}{32,178,170}
\colorlet{coverlinecolor}{customcolor}
\usepackage{cprotect}

\addbibresource[location=local]{reference.bib} % 参考文献,不要删除
\usepackage{listings}         % 导入listings宏包
\usepackage{xcolor}           % 支持颜色

% 配置C++代码样式
\lstset{
    language=C++,             % 语言设置为C++
    basicstyle=\ttfamily,      % 基本样式
    keywordstyle=\color{blue}, % 关键词颜色
    commentstyle=\color{green},% 注释颜色
    stringstyle=\color{red},   % 字符串颜色
    numbers=left,              % 显示行号
    numberstyle=\tiny,         % 行号样式
    stepnumber=1,              % 每行显示行号
    breaklines=true,           % 自动换行
    frame=lines                % 代码块边框样式
}
\begin{document}

\maketitle
\frontmatter

\tableofcontents

\mainmatter

\section{操作系统基本概念}
\begin{enumerate}
  \item 操作系统是对( )进行管理的软件。\\
  A. 软件 \quad B. 硬件 \quad C. 计算机资源 \quad D. 应用程序

  \item 下面的( )资源不是操作系统应该管理的。\\
  A. CPU \quad B. 内存 \quad C. 外存 \quad D. 源程序

  \item 下列选项中,( )不是操作系统关心的问题。\\
  A. 管理计算机裸机 \quad B. 设计、提供用户程序与硬件系统的界面\\
  C. 管理计算机系统资源 \quad D. 高级程序设计语言的编译器

  \item 操作系统的基本功能是( )。\\
  A. 提供功能强大的网络管理工具 \quad B. 提供用户界面方便用户使用\\
  C. 提供方便的可视化编辑程序 \quad D. 控制和管理系统内的各种资源

  \item 现代操作系统中最基本的两个特征是( )。\\
  A. 并发和不确定 \quad B. 并发和共享 \quad C. 共享和虚拟 \quad D. 虚拟和不确定

  \item 下列关于并发性的叙述中,正确的是( )。\\
  A. 并发性是指若干事件在同一时刻发生 \quad B. 并发性是指若干事件在不同时刻发生\\
  C. 并发性是指若干事件在同一时间间隔内发生 \quad D. 并发性是指若干事件在不同时间间隔内发生

  \item 用户可以通过( )两种方式来使用计算机。\\
  A. 命令接口和函数 \quad B. 命令接口和系统调用\\
  C. 命令接口和文件管理 \quad D. 设备管理方式和系统调用

  \item 系统调用是由操作系统提供给用户的,它( )。\\
  A. 直接通过键盘交互方式使用 \quad B. 只能通过用户程序间接使用\\
  C. 是命令接口中的命令 \quad D. 与系统的命令一样

  \item 操作系统提供给编程人员的接口是( )。\\
  A. 库函数 \quad B. 高级语言 \quad C. 系统调用 \quad D. 子程序

  \item 系统调用的目的是( )。\\
  A. 请求系统服务 \quad B. 系统服务 \quad C. 申请系统资源 \quad D. 释放系统资源
  \item 为了方便用户直接或间接地控制自己的作业,操作系统向用户提供了命令接口,该接口又可进一步分为( )。\\
  A. 联机用户接口和脱机用户接口 \quad B. 程序接口和图形接口\\
  C. 联机用户接口和程序接口 \quad D. 脱机用户接口和图形接口

  \item 以下关于操作系统的叙述中,错误的是( )。\\
  A. 操作系统是管理资源的程序 \quad B. 操作系统是管理用户程序执行的程序\\
  C. 操作系统是能使系统资源提高效率的程序 \quad D. 操作系统是用来编程的程序

  \item 提高单机资源利用率的关键技术是( )。\\
  A. 脱机技术 \quad B. 虚拟技术 \quad C. 交换技术 \quad D. 多道程序设计技术

  \item 批处理系统的主要缺点是( )。\\
  A. 系统吞吐量小 \quad B. CPU利用率不高 \quad C. 资源利用率低 \quad D. 无交互能力

  \item 下列选项中,不属于多道程序设计的基本特征的是( )。\\
  A. 制约性 \quad B. 间断性 \quad C. 顺序性 \quad D. 共享性
  \item 操作系统的基本类型主要有( )。\\
  A. 批处理操作系统、分时操作系统和多任务系统 \quad 
  B. 批处理操作系统、分时操作系统和实时操作系统\\
  C. 单用户系统、多用户系统和批处理操作系统 \quad 
  D. 实时操作系统、分时操作系统和多用户系统

  \item 实时操作系统必须在( )内处理来自外部的事件。\\
  A. 一个机器周期 \quad B. 被控制对象规定时间\\
  C. 周转时间 \quad D. 时间片

  \item ( )不是设计实时操作系统的主要追求目标。\\
  A. 安全可靠 \quad B. 资源利用率 \quad C. 及时响应 \quad D. 快速处理

  \item 下列( )应用工作最好采用实时操作系统平台。\\
  I. 航空订票 \quad II. 办公自动化 \quad III. 机床控制\\
  IV. AutoCAD \quad V. 工资管理系统 \quad VI. 股票交易系统\\
  A. I、II 和 III \quad B. I、III 和 IV \quad C. I、V 和 IV \quad D. I、III 和 VI

  \item 下列关于分时系统的叙述中,错误的是( )。\\
  A. 分时系统主要用于批处理作业\\
  B. 分时系统中每个任务依次轮流使用时间片\\
  C. 分时系统的响应时间好\\
  D. 分时系统是一种多用户操作系统
  \item 分时系统的一个重要性能是系统的响应时间,对操作系统的( )因素进行改进有利于改善系统的响应时间。\\
  A. 加大时间片 \quad B. 采用静态页式管理\\
  C. 优先级+非抢占式调度算法 \quad D. 代码可重入

  \item 分时系统追求的目标是( )。\\
  A. 充分利用 I/O 设备 \quad B. 比较快速响应用户\\
  C. 提高系统吞吐率 \quad D. 充分利用内存

  \item 在分时系统中,时间片一定时,( )响应时间越长。\\
  A. 内存越多 \quad B. 内存越少 \quad C. 用户数越多 \quad D. 用户数越少

  \item 在分时系统中,为使多个进程能够及时与系统交互,关键的问题是能在短时间内,使所有就绪进程都能运行。当就绪进程数为100时,为保证响应时间不超过2s,此时的时间片最大应为( )。\\
  A. 10ms \quad B. 20ms \quad C. 50ms \quad D. 100ms

  \item 操作系统有多种类型。允许多个用户以交互的方式使用计算机的操作系统,称为();允许多个用户将若干作业提交给计算机系统集中处理的操作系统,称为();在( )的控制下,计算机系统能及时处理由过程控制反馈的数据,并及时做出响应;在 IBM-PC 中,操作系统称为()。\\
  A. 批处理系统 \quad B. 分时操作系统\\
  C. 实时操作系统 \quad D. 微型计算机操作系统

  \item 下列各种系统中,( )可以使多个进程并行执行。\\
  A. 分时系统 \quad B. 多处理器系统 \quad C. 批处理系统 \quad D. 实时系统

  \item 下列关于操作系统的叙述中,正确的是( )。\\
  A. 批处理操作系统必须在响应时间内处理完一个任务\\
  B. 实时操作系统须在规定时间内处理完来自外部的事件\\
  C. 分时操作系统必须在周转时间内处理完来自外部的事件\\
  D. 分时操作系统必须在调度时间内处理完来自外部的事件

  \item 引入多道程序技术的前提条件之一是系统具有( )。\\
  A. 多个CPU \quad B. 多个终端 \quad C. 中断功能 \quad D. 分时功能

  \item 【2016统考真题】下列关于批处理系统的叙述中,正确的是( )。\\
  I. 批处理系统允许多个用户与计算机直接交互\\
  II. 批处理系统分为单道批处理系统和多道批处理系统\\
  III. 中断技术使得多道批处理系统的 I/O 设备可与 CPU 并行工作\\
  A. 仅Ⅱ、Ⅲ \quad B. 仅Ⅱ \quad C. 仅Ⅰ、Ⅱ \quad D. 仅Ⅰ、Ⅲ

  \item 【2017统考真题】与单道程序系统相比,多道程序系统的优点是( )。\\
  I. CPU利用率高 \quad II. 系统开销小\\
  III. 系统吞吐量大 \quad IV. I/O设备利用率高\\
  A. 仅Ⅰ、Ⅲ \quad B. 仅Ⅰ、IV \quad C. 仅Ⅱ、Ⅲ \quad D. 仅Ⅰ、Ⅲ、IV
  \item 【2018统考真题】下列关于多任务操作系统的叙述中,正确的是( )。\\
  I. 具有并发和并行的特点\\
  II. 需要实现对共享资源的保护\\
  III. 需要运行在多 CPU 的硬件平台上\\
  A. 仅 I \quad B. 仅 II \quad C. 仅 I、II \quad D. I、II、III

  \item 【2022统考真题】下列关于多道程序系统的叙述中,不正确的是( )。\\
  A. 支持进程的并发执行 \quad B. 不必支持虚拟存储管理\\
  C. 需要实现对共享资源的管理 \quad D. 进程数越多 CPU 利用率越高

  \item 【2013统考真题】下列关于操作系统的说法中,错误的是( )。\\
  I. 在通用操作系统管理下的计算机上运行程序,需要向操作系统预订运行时间\\
  II. 在通用操作系统管理下的计算机上运行程序,需要确定起始地址,并从这个地址开始执行\\
  III. 操作系统需要提供高级程序设计语言的编译器\\
  IV. 管理计算机系统资源是操作系统关心的主要问题\\
  A. I \quad B. I、II \quad C. I、II、III \quad D. I、II、III、IV

  \item 下列说法中,正确的是( )。\\
  I. 批处理的主要缺点是需要大量内存\\
  II. 当计算机提供了核心态和用户态时,输入/输出指令必须在核心态下执行\\
  III. 操作系统中采用多道程序设计技术的最主要原因是提高 CPU 和外部设备的可靠性\\
  IV. 操作系统中,通道技术是一种硬件技术\\
  A. I、II \quad B. I、III \quad C. II、IV \quad D. II、III、IV

  \item 下列关于系统调用的说法中,正确的是( )。\\
  I. 用户程序使用系统调用命令,该命令经过编译后形成若干参数和陷入指令\\
  II. 用户程序使用系统调用命令,该命令经过编译后形成若干参数和屏蔽中断指令\\
  III. 用户程序创建一个新进程,需使用操作系统提供的系统调用接口\\
  IV. 当操作系统完成用户请求的系统调用功能后,应使 CPU 从内核态转到用户态\\
  A. I、III \quad B. III、IV \quad C. I、III、IV \quad D. II、III、IV

  \item ( )是操作系统必须提供的功能。\\
  A. 图形用户界面 (GUI) \quad B. 为进程提供系统调用命令\\
  C. 中断处理 \quad D. 编译源程序

  \item 用户程序在用户态下要使用特权指令引起的中断属于( )。\\
  A. 故障异常 \quad B. 终止异常 \quad C. 外部中断 \quad D. 陷入中断

  \item 处理器执行的指令被分为两类,其中有一类称为特权指令,它只允许( )使用。\\
  A. 操作员 \quad B. 联机用户 \quad C. 目标程序 \quad D. 操作系统

  \item 在中断发生后,进入中断处理的程序属于( )。\\
  A. 用户程序 \quad B. 可能是用户程序,也可能是 OS 程序\\
  C. 操作系统程序 \quad D. 单独的程序,既不是用户程序也不是 OS 程序

  \item 计算机区分核心态和用户态指令后,从核心态到用户态的转换是由操作系统程序执行后完成的,而用户态到核心态的转换则是由( )完成的。\\
  A. 硬件 \quad B. 核心态程序 \quad C. 用户程序 \quad D. 中断处理程序

  item 可在用户态执行的指令是 (    )。\\
    A. 屏蔽中断 \quad B. 设置时钟的值 \quad C. 修改内存单元的值 \quad D. 停机

    \item 在操作系统中,只能在核心态下运行的指令是 (    )。\\
    A. 读时钟指令 \quad B. 置时钟指令 \quad C. 取数指令 \quad D. 寄存器清零

    \item 下列程序中,不工作在内核态的是 (    )。\\
    A. 命令解释程序 \quad B. 磁盘调度程序 \quad C. 中断处理程序 \quad D. 进程调度程序

    \item “访管”指令 (    ) 使用。\\
    A. 仅在用户态下 \quad B. 仅在核心态下 \quad C. 在规定时间内 \quad D. 在调度时间内

    \item 当 CPU 执行操作系统代码时,处理器处于 (    )。\\
    A. 自由态 \quad B. 用户态 \quad C. 核心态 \quad D. 就绪态

    \item 在操作系统中,只能在核心态下执行的指令是 (    )。\\
    A. 读时钟 \quad B. 取数 \quad C. 系统调用命令 \quad D. 寄存器清“0”

    \item 下列选项中,必须在核心态下执行的指令是 (    )。\\
    A. 从内存中取数 \quad B. 将运算结果装入内存\\
    C. 算术运算 \quad D. 输入/输出

    \item CPU 处于核心态时,它可以执行的指令是 (    )。\\
    A. 只有特权指令 \quad B. 只有非特权指令\\
    C. 只有“访管”指令 \quad D. 除“访管”指令的全部指令

    \item (    ) 程序可执行特权指令。\\
    A. 同组用户 \quad B. 操作系统 \quad C. 特权用户 \quad D. 一般用户

    \item 下列中断事件中,能引起外部中断的事件是 (    )。\\
    I. 时钟中断 \quad II. 访管中断 \quad III. 缺页中断\\
    A. I \quad B. III \quad C. I 和 II \quad D. II 和 III

    \item 下列关于库函数和系统调用的说法中,不正确的是 (    )。\\
    A. 库函数运行在用户态,系统调用运行在内核态\\
    B. 使用库函数时开销较小,使用系统调用时开销较大\\
    C. 库函数不方便替换,系统调用通常很方便被替换\\
    D. 库函数可以很方便地调试,而系统调用很麻烦

    \item 下列关于系统调用和一般过程调用的说法中,正确的是 (    )。\\
    A. 两者都需要将当前 CPU 中的 PSW 和 PC 的值压栈,以保存现场信息\\
    B. 系统调用的被调用过程一定运行在内核态\\
    C. 一般过程调用的被调用过程一定运行在用户态\\
    D. 两者的调用过程与被调用过程一定都运行在用户态

    \item 用户在程序中试图读某文件的第 100 个逻辑块,使用操作系统提供的 (    ) 接口。\\
    A. 系统调用 \quad B. 键盘命令 \quad C. 原语 \quad D. 图形用户接口

    \item 【2011 统考真题】下列选项中,在用户态执行的是 (    )。\\
    A. 命令解释程序 \quad B. 缺页处理程序\\
    C. 进程调度程序 \quad D. 时钟中断处理程序

    \item 【2012 统考真题】下列选项中,不可能在用户态发生的事件是 (    )。\\
    A. 系统调用 \quad B. 外部中断 \quad C. 进程切换 \quad D. 缺页

    \item 【2012 统考真题】中断处理和子程序调用都需要压栈,以便保护现场,中断处理一定会保存而子程序调用不需要保存其内容的是 (    )。\\
    A. 程序计数器 \quad B. 程序状态字寄存器\\
    C. 通用数据寄存器 \quad D. 通用地址寄存器

    \item 【2013 统考真题】下列选项中,会导致用户进程从用户态切换到内核态的操作是 (    )。\\
    I. 整数除以零 \quad II. sin() 函数调用 \quad III. read 系统调用\\
    A. 仅 I、II \quad B. 仅 I、III \quad C. 仅 II、III \quad D. I、II 和 III

    \item 【2014 统考真题】下列指令中,不能在用户态执行的是 (    )。\\
    A. trap 指令 \quad B. 跳转指令 \quad C. 压栈指令 \quad D. 关中断指令

    \item 【2015 统考真题】处理外部中断时,应该由操作系统保存的是 (    )。\\
    A. 程序计数器 (PC) 的内容 \quad B. 通用寄存器的内容\\
    C. 块表 (TLB) 中的内容 \quad D. Cache 中的内容

    \item 【2015 统考真题】假定下列指令已装入指令寄存器,则执行时不可能导致 CPU 从用户态变为内核态 (系统态) 的是 (    )。\\
    A. DIV R0, R1 \quad B. INT n \quad C. NOT R0 \quad D. MOV R0, addr

    \item 【2021统考真题】下列指令中,只能在内核态执行的是 (    )。\\
    A. trap指令 \quad B. I/O 指令 \quad C. 数据传送指令 \quad D. 设置断点指令

    \item 【2021统考真题】下列选项中,通过系统调用完成的操作是 (    )。\\
    A. 页置换 \quad B. 进程调度 \quad C. 创建新进程 \quad D. 生成随机整数

    \item 【2022统考真题】下列关于CPU模式的叙述中,正确的是 (    )。\\
    A. CPU处于用户态时只能执行特权指令\\
    B. CPU处于内核态时只能执行特权指令\\
    C. CPU处于用户态时只能执行非特权指令\\
    D. CPU处于内核态时只能执行非特权指令

    \item 【2022统考真题】执行系统调用的过程涉及下列操作,其中由操作系统完成的是 (    )。\\
    I. 保存断点和程序状态字\\
    II. 保存通用寄存器的内容\\
    III. 执行系统调用服务例程\\
    IV. 将CPU模式改为内核态\\
    A. 仅 I、III \quad B. 仅 II、III \quad C. 仅 II、IV \quad D. 仅 II、III、IV

    \item 【2023统考真题】在操作系统内核中,中断向量表适合采用的数据结构是 (    )。\\
    A. 数组 \quad B. 队列 \quad C. 单向链表 \quad D. 双向链表

    \item 用 (    ) 设计的操作系统结构清晰且便于调试。\\
    A. 分层式构架 \quad B. 模块化构架 \quad C. 微内核构架 \quad D. 宏内核构架

    \item 下列关于分层式结构操作系统的说法中,(    ) 是错误的。\\
    A. 各层之间只能是单向依赖或单向调用\\
    B. 容易实现在系统中增加或替换一层而不影响其他层\\
    C. 具有非常灵活的依赖关系\\
    D. 系统效率较低

    \item 在操作系统结构设计中,层次结构的操作系统最显著的不足是 (    )。\\
    A. 不能访问更低的层次 \quad B. 太复杂且效率低\\
    C. 设计困难 \quad D. 模块太少

    \item 下列选项中,(    ) 不属于模块化操作系统的特点。\\
    A. 很多模块化的操作系统,可以支持动态加载新模块到内核,适应性强\\
    B. 内核中的某个功能模块出错不会导致整个系统崩溃,可靠性高\\
    C. 内核中的各个模块,可以相互调用,无须通过消息传递进行通信,效率高\\
    D. 各模块间相互依赖,相比于分层式操作系统,模块化操作系统更难调试和验证

    \item 相对于微内核系统,(    ) 不属于大内核操作系统的缺点。\\
    A. 占用内存空间大 \quad B. 缺乏可扩展性而不方便移植\\
    C. 内核切换太慢 \quad D. 可靠性较低

    \item 下列说法中,(    ) 不适合描述微内核操作系统。\\
    A. 内核足够小 \quad B. 功能分层设计\\
    C. 基于C/S模式 \quad D. 策略与机制分离

    \item 对于以下五种服务,在采用微内核结构的操作系统中,(    ) 不宜放在微内核中。\\
    I. 进程间通信机制 \quad II. 低级I/O \quad III. 低级进程管理和调度\\
    IV. 中断和陷入处理 \quad V. 文件系统服务\\
    A. I、II 和 III \quad B. II 和 V \quad C. 仅 V \quad D. IV 和 V

    \item 相对于传统操作系统结构,采用微内核结构设计和实现操作系统有诸多好处,下列 (    ) 是微内核结构的特点。\\
    I. 使系统更高效\\
    II. 添加系统服务时,不必修改内核\\
    III. 微内核结构没有单一内核稳定\\
    IV. 使系统更可靠\\
    A. I、III、IV \quad B. I、II、IV \quad C. I、IV \quad D. I、IV

    \item 下列关于操作系统结构的说法中,正确的是 (    )。\\
    I. 当前广泛使用的Windows 操作系统,采用的是分层式OS结构\\
    II. 模块化的OS结构设计的基本原则是,每一层都仅使用其底层所提供的功能和服务,这样就使系统的调试和验证都变得容易\\
    III. 由于微内核结构能有效支持多处理机运行,故非常适合于分布式系统环境\\
    IV. 采用微内核结构设计和实现操作系统具有诸多好处,如添加系统服务时,不必修改内核、使系统更高效。\\
    A. I 和 II \quad B. I 和 III \quad C. III \quad D. III 和 IV

    \item 下列关于微内核操作系统的描述中,不正确的是 (    )。\\
    A. 可增加操作系统的可靠性\\
    B. 可提高操作系统的执行效率\\
    C. 可提高操作系统的可移植性\\
    D. 可提高操作系统的可拓展性
    \item 下列关于操作系统外核 (exokernel) 的说法中,错误的是 (    )。\\
    A. 外核可以给用户进程分配未经抽象的硬件资源\\
    B. 用户进程通过调用“库”请求操作系统外核的服务\\
    C. 外核负责完成进程调度\\
    D. 外核可以减少虚拟硬件资源的“映射”开销,提升系统效率

    \item 对于计算机操作系统引导,描述不正确的是 (    )。\\
    A. 计算机的引导程序驻留在 ROM 中,开机后自动执行\\
    B. 引导程序先做关键部位的自检,并识别已连接的外设\\
    C. 引导程序会将硬盘中存储的操作系统全部加载到内存中\\
    D. 若计算机中安装了双系统,引导程序会与用户交互加载有关系统

    \item 存放操作系统自举程序的芯片是 (    )。\\
    A. SRAM \quad B. DRAM \quad C. ROM \quad D. CMOS

    \item 计算机操作系统的引导程序位于 (    ) 中。\\
    A. 主板 BIOS \quad B. 片外 Cache \quad C. 主存 ROM 区 \quad D. 硬盘

    \item 计算机的启动过程是 (    )。\\
    ① CPU 加电,CS:IP 指向 FFFF0H\\
    ② 进行操作系统引导\\
    ③ 执行 JMP 指令跳转到 BIOS\\
    ④ 登记 BIOS 中断程序入口地址\\
    ⑤ 硬件自检\\
    A. ①②③④⑤ \quad B. ①③⑤④② \quad C. ①③④⑤② \quad D. ①⑤③④②

    \item 检查分区表是否正确,确定哪个分区为活动分区,并在程序结束时将该分区的启动程序(操作系统引导扇区)调入内存加以执行,这是 (    ) 的任务。\\
    A. MBR \quad B. 引导程序 \quad C. 操作系统 \quad D. BIOS

    \item 下列关于虚拟机的说法中,正确的是 (    )。\\
    I. 虚拟机可以用软件实现\\
    II. 虚拟机可以用硬件实现\\
    III. 多台虚拟机可同时运行在同一物理机器上,它实现了真正的并行\\
    A. I 和 II \quad B. I 和 III \quad C. 仅 I \quad D. I、II 和 III

    \item 下列关于 VMware Workstation 虚拟机的说法中,错误的是 (    )。\\
    A. 真实硬件不会直接执行虚拟机中的敏感指令\\
    B. 虚拟机中只能安装一种操作系统\\
    C. 虚拟机是运行在计算机中的一个应用程序\\
    D. 虚拟机文件封装在一个文件夹中,并存储在数据存储器中

    \item 虚拟机的实现离不开虚拟机管理程序 (VMM),下列关于 VMM 的说法中正确的是 (    )。\\
    I. 第一类 VMM 直接运行在硬件上,其效率通常高于第二类 VMM\\
    II. 由于 VMM 的上层需要支持操作系统的运行、应用程序的运行,因此实现 VMM 的代码量通常大于实现一个完整操作系统的代码量\\
    III. VMM 可将一台物理机器虚拟化为多台虚拟机器\\
    IV. 为了支持客户操作系统的运行,第二类 VMM 需要完全运行在最高特权级\\
    A. I、II 和 III \quad B. I 和 III \quad C. I、II、III 和 IV \quad D. I、III 和 IV

    \item 【2013统考真题】计算机开机后,操作系统最终被加载到 (    )。\\
    A. BIOS \quad B. ROM \quad C. EPROM \quad D. RAM

    \item 【2022统考真题】下列选项中,需要在操作系统进行初始化过程中创建的是 (    )。\\
    A. 中断向量表 \quad B. 文件系统的根目录\\
    C. 硬盘分区表 \quad D. 文件系统的索引节点表
    \item 【2023统考真题】与宏内核操作系统相比,下列特征中,微内核操作系统具有的是 (    )。\\
  I. 较好的性能 \quad II. 较高的可靠性 \quad III. 较高的安全性 \quad IV. 较强的可扩展性\\
  A. 仅 II、IV \quad B. 仅 I、II、I \quad C. 仅 I、III、IV \quad D. 仅 II、III、IV
\end{enumerate}


\end{document}
