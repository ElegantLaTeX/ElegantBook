\documentclass[lang=cn,12pt]{elegantbook}

\title{ElegantBook:优美的 \LaTeX{} 书籍模板}
\subtitle{Elegant\LaTeX{} 经典之作}

\author{Ethan Deng \& Liam Huang}
\institute{Elegant\LaTeX{} Program}
\date{April 9, 2022}
\version{4.3}
\bioinfo{自定义}{信息}

\extrainfo{不要以为抹消过去,重新来过,即可发生什么改变。—— 比企谷八幡}

\setcounter{tocdepth}{3}

\logo{logo-blue.png}
\cover{cover.jpg}

% 本文档命令
\usepackage{array}
\newcommand{\ccr}[1]{\makecell{{\color{#1}\rule{1cm}{1cm}}}}

% 修改标题页的橙色带
% \definecolor{customcolor}{RGB}{32,178,170}
% \colorlet{coverlinecolor}{customcolor}

\renewcommand\appendix{\setcounter{secnumdepth}{-2}}

\begin{document}

%\maketitle
%\frontmatter

%\tableofcontents

%\mainmatter



\section*{2022.06.06}

设 $x_{n} = \left(1+\dfrac{1}{n^2}\right)\left(1+\dfrac{2}{n^2}\right)
  \cdots \left(1+\dfrac{n}{n^2}\right),$
则极限 $\lim\limits_{n\to\infty} x_{n}$ = \_\_\_\_.
\\  \\

\begin{solution}
  法一: 由于 $\dfrac{x}{1+x}<\ln\left(1+x\right)<x,$ 故 
  $x-\ln\left(1+x\right)<\dfrac{x^2}{1+x}.$
  因此
  $$\left|\sum_{k=1}^{n}\ln \left(1+\dfrac{k}{n^2}\right) - 
  \sum_{k=1}^{n}\dfrac{k}{n^2}\right| <
  \sum_{k=1}^{n}\dfrac{\left(\dfrac{k}{n^2}\right)^2}{
    1+\dfrac{k}{n^2}}<\sum_{k=1}^{n} \left(\dfrac{n}{n^2}\right)^2 =
    \dfrac{1}{n} \to 0 (n\to\infty)
  $$
  故由夹逼准则得: $$\lim\limits_{n\to\infty} \ln \left(1+\dfrac{k}{n^2}\right) =
    \lim\limits_{n\to\infty} \sum_{k=1}^{n}\dfrac{k}{n^2} =
    \lim\limits_{n\to\infty} \dfrac{n+1}{2n} = \dfrac{1}{2}.$$
  故 $$\lim\limits_{n\to\infty} x_{n} = 
    \exp\left\{\lim\limits_{n\to\infty} \ln \left(1+\dfrac{k}{n^2}\right)\right\} =
    \sqrt{e}.$$
  \vspace{6pt}

  法二: 因
  $$
  x_{n} = \exp\left\{ \sum_{k=1}^{n} \ln \left(1+\dfrac{k}{n^2}\right)\right\} = 
  \exp\left\{ \sum_{k=1}^{n} \left[\dfrac{k}{n^2}+\mathcal{O}\left(
    \dfrac{k^2}{n^4}\right) \right] \right\} =
  \exp\left\{ \dfrac{1}{2} + \mathcal{O}\left(\dfrac{1}{n}\right) \right\},
  $$
  故 $\lim\limits_{n\to\infty} x_{n} = \sqrt{e}.$

\end{solution}

\newpage

\section*{2022.06.07}

设 $x_{0} = 1, x_{n} = \dfrac{1+2x_{n-1}}{1+x_{n-1}}, n = 1, 2, \cdots.$
证明数列 $\{x_{n}\}$ 收敛, 并求极限 $\lim\limits_{n\to\infty} x_{n}.$
\\ \\

\begin{solution}
  法一: $k=0$ 时, 有 $1 \leqslant x_{0} < 2$ 成立.
  假设 $k=n\in \mathbb{N}$ 时, 有 $1 \leqslant x_{n} < 2,$ 则
  $k=n+1$ 时, 有 
    $$1 \leqslant 1+\dfrac{x_{n}}{1+x_{n}} = x_{n+1} = 2-\dfrac{1}{1+x_{n}} < 2,$$
  故由数学归纳法知 $1 \leqslant x_{n} < 2.$ 又 
  $$ x_{n+1} - x_{n}
    =\dfrac{1+2x_{n}}{1+x_{n}} - \dfrac{1+2x_{n-1}}{1+x_{n-1}}
    =\dfrac{x_{n}-x_{n-1}}{\left(1+x_{n}\right)\left(1+x_{n-1}\right)}>0,$$
  且 $x_{1}-x_{0} = \dfrac{1}{2} > 0,$ 知数列 $\{x_{n}\}$ 单调增加. 由单调有界准则得:
  数列 $\{x_{n}\}$ 收敛, 令 $A=\lim\limits_{n\to\infty} x_{n}\geqslant 1,$ 并对等式
  $x_{n} = \dfrac{1+2x_{n-1}}{1+x_{n-1}}$ 两边同时取极限, 得 $A=\dfrac{1+2A}{1+A}$
  即 $A^2-A-1=0.$ 解得: $A = \dfrac{\sqrt{5} + 1}{2}$ 或 
  $\dfrac{1-\sqrt{5}}{2}$ (舍去).
  \vspace{6pt}

  法二: 由于 $x_{n} = \dfrac{1+2x_{n-1}}{1+x_{n-1}}$ 满足 $x_{n} = f(x_{n-1})$
  这种形式, 我们尝试令 $f(x) = \dfrac{1+2x}{1+x},$ 由于
  $f'(x) = \dfrac{1}{(1+x)^2} > 0,$ 且  $f'(x) = \dfrac{1}{(1+x)^2} > 0,$ 知
  数列 $\{x_{n}\}$ 单调增加. 其他过程与法一类似.
  \vspace{6pt}

  法三: 由于 $$\begin{aligned} &\quad\ 
    \left| x_{n+1} - \dfrac{\sqrt{5}+1}{2} \right| = 
    \left| \dfrac{1+2x_{n}}{1+x_{n}} - \dfrac{\sqrt{5}+1}{2} \right| =
    \dfrac{\left|\left(3-\sqrt{5}\right)x_{n}+1-\sqrt{5}\right|}{2\left(1+x_{n}\right)} \\ &=
    \dfrac{2}{(3+\sqrt{5})\left(1+x_{n}\right)}\left|x_{n}-\dfrac{\sqrt{5}+1}{2}\right| <
    \dfrac{2}{3+\sqrt{5}}\left|x_{n}-\dfrac{\sqrt{5}+1}{2}\right| <
    \left(\dfrac{2}{3+\sqrt{5}}\right)^2\left|x_{n-1}-\dfrac{\sqrt{5}+1}{2}\right| \\ & < \cdots <
    \left(\dfrac{2}{3+\sqrt{5}}\right)^{n+1}\left|x_{0}-\dfrac{\sqrt{5}+1}{2}\right| \to 0 (n\to\infty)
    , \text{故由夹逼准则得到}: \lim\limits_{n\to\infty} x_{n} = \dfrac{\sqrt{5}+1}{2} .
    \end{aligned}$$
  \vspace{6pt}

  法四: 由斐波那契数列 $\{f_{n}\}$ 与数学归纳法可证得 
  $x_{n} = \dfrac{f_{2n+2}}{f_{2n+1}}.$
  结合数列 $\{f_{n}\}$ 的通项公式 $$f_{n} = 
    \dfrac{1}{\sqrt{5}}\left[\left(\dfrac{1+\sqrt{5}}{2}\right)^n-
      \left(\dfrac{1-\sqrt{5}}{2}\right)^n\right]$$ 得到:
  $$x_{n} = \dfrac{f_{2n+2}}{f_{2n+1}}=
    \dfrac{\left(\dfrac{1+\sqrt{5}}{2}\right)^{2n+2}-
      \left(\dfrac{1-\sqrt{5}}{2}\right)^{2n+2}}{
        \left(\dfrac{1+\sqrt{5}}{2}\right)^{2n+1}-
          \left(\dfrac{1-\sqrt{5}}{2}\right)^{2n+1}}=
    \dfrac{\dfrac{1+\sqrt{5}}{2}-\dfrac{\sqrt{5}-1}{2}
      \left(\dfrac{\sqrt{5}-1}{\sqrt{5}+1}\right)^{2n+1}}{
        1+\left(\dfrac{\sqrt{5}-1}{\sqrt{5}+1}\right)^{2n+1}}.$$
  显然 $\lim\limits_{n\to\infty} \left(\dfrac{\sqrt{5}-1}{\sqrt{5}+1}\right)^{2n+1} = 0,$
  等式两边同时取极限得到: $\lim\limits_{n\to\infty} x_{n} = \dfrac{\sqrt{5}+1}{2}.$
\end{solution}

\newpage

\section*{2022.06.08}

设 $2x_{1} = 1, 2x_{n} = 1-x_{n}^2, n = 1, 2, \cdots.$
证明数列 $\{x_{n}\}$ 收敛, 并求极限 $\lim\limits_{n\to\infty} x_{n}.$
\\ \\

\begin{solution}
  因 $0 < x_{1} = \dfrac{1}{2} < 1,$ 假设 $k=n\in\mathbb{N}$ 时,
  $0 < x_{n} < 1$ 成立. 则 $0 < x_{n+1} = \dfrac{1-x_{n}^2}{2} < 1.$
  由数学归纳法知 $0 < x_{n} < 1, n = 1, 2, \cdots.$ 故
  $$\begin{aligned}
    \left| x_{n+1} - \left(\sqrt{2} - 1\right)\right| &=
    \left| \dfrac{1-x_{n}^2}{2} - \sqrt{2} + 1\right| = 
    \dfrac{x_{n}^2 + 2\sqrt{2} - 3}{2} \\ &= 
    \dfrac{x_{n} + \sqrt{2} - 1}{2}\left| x_{n} - \left(\sqrt{2} - 1\right)\right| \\ &<
    \dfrac{1}{\sqrt{2}}\left| x_{n} - \left(\sqrt{2} - 1\right)\right| \\ &<
    \left(\dfrac{1}{\sqrt{2}}\right)^2\left| x_{n-1} - \left(\sqrt{2} - 1\right)\right| < \cdots \\ & <
    \left(\dfrac{1}{\sqrt{2}}\right)^n\left| x_{1} - \left(\sqrt{2} - 1\right)\right| \to 0 (n\to\infty).
  \end{aligned}
  $$
  故由夹逼准则得到: $\lim\limits_{n\to\infty} x_{n} = \sqrt{2} - 1.$
\end{solution}

\newpage

\section*{2022.06.09}

设 $f(x) = 1 - \cos x,$ 则 $\lim\limits_{x \to 0} 
  \dfrac{\left(1-\sqrt{\cos x}\right)\left(1-\sqrt[3]{\cos x}\right)
  \left(1-\sqrt[4]{\cos x}\right)\left(1-\sqrt[5]{\cos x}\right)}
  {f\left\{f\left[f\left(x\right)\right]\right\}}=\_\_\_\_.$
\\ \\

\begin{solution}
  $\dfrac{1}{15}.$ 
  因 $$\begin{cases}
    &1- \left(\cos x\right)^a = 1 - \left[1-\left(1-\cos x\right)\right]^a 
      \sim a(1-\cos x) \sim \dfrac{ax^2}{2},\\
    &f\left\{f\left[f\left(x\right)\right]\right\} \sim 
      \dfrac{f^2\left[f\left(x\right)\right]}{2} \sim
      \dfrac{\left[\dfrac{f^2\left(x\right)}{2}\right]^2}{2}\sim 
      \dfrac{1}{8}\left(\dfrac{x^2}{2}\right)^4 = \dfrac{x^8}{128}.
  \end{cases}$$

  故 $$\begin{aligned}
    &\quad \ \lim\limits_{x \to 0} 
      \dfrac{\left(1-\sqrt{\cos x}\right)\left(1-\sqrt[3]{\cos x}\right)
      \left(1-\sqrt[4]{\cos x}\right)\left(1-\sqrt[5]{\cos x}\right)}
      {f\left\{f\left[f\left(x\right)\right]\right\}} \\
    &= \lim\limits_{x \to 0} \dfrac{\left(\dfrac{x^2}{4}\right)\left(\dfrac{x^2}{6}\right)
    \left(\dfrac{x^2}{8}\right)\left(\dfrac{x^2}{10}\right)}
    {\dfrac{x^8}{128}}=\dfrac{1}{15}.
  \end{aligned}$$

\end{solution}


\section*{2022.06.10}

\end{document}
