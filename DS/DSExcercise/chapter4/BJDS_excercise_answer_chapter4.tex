\documentclass[lang=cn,newtx,10pt,scheme=chinese]{../../../elegantbook}

\title{基础提高练习题}
\subtitle{北街学长倾力之作}

\author{北街}
% \institute{Elegant\LaTeX{} Program}
\date{2022/12/31}
\version{1.0}
% \bioinfo{自定义}{信息}

% \extrainfo{注意:本模板自 2023 年 1 月 122222 日开始,不再更新和维护!}

\setcounter{tocdepth}{3}

\logo{../../figure/logo-blue.png}
\cover{../../figure/cover.jpg}

% 本文档命令
\usepackage{array}
\newcommand{\ccr}[1]{\makecell{{\color{\#1}\rule{1cm}{1cm}}}}

% 修改标题页的橙色带
\definecolor{customcolor}{RGB}{32,178,170}
\colorlet{coverlinecolor}{customcolor}
\usepackage{cprotect}

\addbibresource[location=local]{reference.bib} % 参考文献,不要删除
\usepackage{listings}         % 导入listings宏包
\usepackage{xcolor}           % 支持颜色

% 配置C++代码样式
\lstset{
    language=C++,             % 语言设置为C++
    basicstyle=\ttfamily,      % 基本样式
    keywordstyle=\color{blue}, % 关键词颜色
    commentstyle=\color{green},% 注释颜色
    stringstyle=\color{red},   % 字符串颜色
    numbers=left,              % 显示行号
    numberstyle=\tiny,         % 行号样式
    stepnumber=1,              % 每行显示行号
    breaklines=true,           % 自动换行
    frame=lines                % 代码块边框样式
}
\begin{document}

\maketitle
\frontmatter

\tableofcontents

\mainmatter

\chapter{串}
\begin{enumerate}

    \item 设主串 $S = \texttt{"abaabaabcabaabc"}$,模式串 $P = \texttt{"abaabc"}$,采用 KMP 算法进行模式匹配,到匹配成功时为止,在匹配过程中进行的单个字符间的比较次数是( )。  
    【2019 年全国试题 9(2 分)】

    A. 9 \quad B. 10 \quad C. 12 \quad D. 15  

    答案:\textcolor{red}{C}

    解析:\\
        在 KMP 算法中,模式匹配的核心是利用部分匹配表(\textbf{\textcolor{blue}{Next 数组}})来减少字符的重复比较。以下是匹配过程的详细分析:
                
                主串 $S = \texttt{"abaabaabcabaabc"}$\\
                模式串 $P = \texttt{"abaabc"}$\\
                \textbf{\textcolor{blue}{Next 数组的计算}}\\
                模式串 $P$ 的 \textbf{\textcolor{blue}{Next 数组}}为:\\
                Next = [-1, 0, 0, 1, 1, 2]\\
                
                \textbf{\textcolor{blue}{匹配过程}}\\
                
                初始时,$i = 0$(主串指针),$j = 0$(模式串指针)。\\
                每次比较 $S[i]$ 和 $P[j]$,如果匹配成功,则 $i$ 和 $j$ 同时加 1;如果失配,则根据 \textbf{\textcolor{blue}{Next 数组}}调整 $j$ 的位置。\\
                \textbf{\textcolor{blue}{匹配步骤:}}\\
                
                比较 $S[0]$ 和 $P[0]$,匹配($i = 1, j = 1$)。\\
                比较 $S[1]$ 和 $P[1]$,匹配($i = 2, j = 2$)。\\
                比较 $S[2]$ 和 $P[2]$,匹配($i = 3, j = 3$)。\\
                比较 $S[3]$ 和 $P[3]$,匹配($i = 4, j = 4$)。\\
                比较 $S[4]$ 和 $P[4]$,匹配($i = 5, j = 5$)。\\
                比较 $S[5]$ 和 $P[5]$,\textbf{\textcolor{red}{失配}}($i = 5, j = 2$,根据 \textbf{\textcolor{blue}{Next 数组}}调整 $j$)。\\
                比较 $S[5]$ 和 $P[2]$,匹配($i = 6, j = 3$)。\\
                比较 $S[6]$ 和 $P[3]$,匹配($i = 7, j = 4$)。\\
                比较 $S[7]$ 和 $P[4]$,匹配($i = 8, j = 5$)。\\
                比较 $S[8]$ 和 $P[5]$,匹配($i = 9, j = 6$,\textbf{\textcolor{green}{匹配成功}})。
                
                \textbf{\textcolor{blue}{比较次数统计}}\\
                
                匹配成功的比较次数:\textbf{\textcolor{green}{9 次}}。\\
                失配后重新比较的次数:\textbf{\textcolor{red}{3 次}}($S[5]$ 和 $P[2]$,$S[6]$ 和 $P[3]$,$S[7]$ 和 $P[4]$)。\\
                \textbf{\textcolor{blue}{总比较次数:$9 + 3 = 12$}}。\\
                因此,匹配过程中进行的单个字符间的比较次数是 \textbf{\textcolor{blue}{12}}。\\

    \item 已知字符串 $S = \texttt{"abaabaabacacaabaabcc"}$,模式串 $P = \texttt{"abaabc"}$,采用 KMP 算法进行匹配,第一次出现“失配”($S[i] \neq P[j]$)时,$i=5$ 且 $j=5$,则下次开始匹配时,$i$ 和 $j$ 的值分别是( )。  
    【2015 年全国试题 8(2 分)】  

    A. $i=1, j=0$ \quad B. $i=5, j=0$ \quad C. $i=5, j=2$ \quad D. $i=6, j=2$  

    答案:\textcolor{red}{D}

    解析:\\
    在 KMP 算法中,当发生失配时,模式串指针 $j$ 的位置会根据 \textbf{\textcolor{blue}{Next 数组}}的值进行调整,而主串指针 $i$ 保持不变或继续向后移动。\\

    \textbf{\textcolor{blue}{已知条件:}}\\
    主串 $S = \texttt{"abaabaabacacaabaabcc"}$。\\
    模式串 $P = \texttt{"abaabc"}$。\\
    失配发生时,$i = 5$ 且 $j = 5$。\\

    \textbf{\textcolor{blue}{计算模式串 $P$ 的 Next 数组:}}\\

    Next = [-1,0, 0, 1, 1, 2]\\

    模式串 $P = \texttt{"abaabc"}$ 的 \textbf{\textcolor{blue}{Next 数组}}为:\\

    \textbf{\textcolor{blue}{匹配过程分析:}}\\
    当前失配位置为 $i = 5$(主串指针),$j = 5$(模式串指针)。\\
    根据 \textbf{\textcolor{blue}{Next 数组}},当 $j = 5$ 时,失配后模式串指针 $j$ 应调整为 $Next[5] = 2$。\\
    主串指针 $i$ 保持不变或向后移动,因此下次匹配从 $i = 6$ 开始。\\

    \textbf{\textcolor{blue}{下次匹配的起始位置:}}\\
    主串指针 $i = 6$。\\
    模式串指针 $j = 2$。\\

    因此,答案是 \textbf{\textcolor{red}{D}}. $i=6, j=2$。

    \item 下面关于串的叙述中,哪一个是不正确的( )。  
    【北方交通大学 2001 一、5(2 分);江苏大学 2005 一、6(2 分)】  

    A. 串是字符的有限序列  

    B. 空串是由空格构成的串  

    C. 模式匹配是串的一种重要运算 

    D. 串既可以采用顺序存储,也可以采用链式存储  

    答案:\textcolor{red}{B}
    解析:\\
    \textbf{\textcolor{blue}{选项分析:}}\\
    A. 正确,串是字符的有限序列。\\
    B. 错误,空串是长度为 0 的串,不是由空格构成的串。\\
    C. 正确,模式匹配是串的一种重要运算。\\
    D. 正确,串既可以采用顺序存储,也可以采用链式存储。\\
    因此,答案是 \textbf{\textcolor{red}{B}}. 空串是由空格构成的串(错误)。\\

    \item 若串 $S_1 = \texttt{"ABCDEFG"}$,$S_2 = \texttt{"9898"}$,$S_3 = \texttt{"\#\#\#"}$,$S_4 = \texttt{"012345"}$,
    执行  
    \texttt{concat(replace(S1, substr(S1, length(S2), length(S3)), S3), substr(S4, index(S2, '8'), length(S2)))}  
    其结果为( )。  
    【北方交通大学 1999 一、5(2 分)】  

    A. \texttt{"ABC\#\#\#{G0123}"} \quad B. \texttt{"ABCD\#\#\#{2345}"}  

    C. \texttt{"ABC\#\#\#{G2345}"} \quad D. \texttt{"ABC\#\#\#{2345}"}  

    E. \texttt{"ABC\#\#\#{G1234}"} \quad F. \texttt{"ABCD\#\#\#{1234}"}  

    G. \texttt{"ABC\#\#{01234}"}

    答案:C. \texttt{"ABC\#\#\#{G2345}"}

    解析:\\
    我们逐步分析题目中函数的执行过程:

    已知:\\
    $S_1 = \texttt{"ABCDEFG"}$\\
    $S_2 = \texttt{"9898"}$\\
    $S_3 = \texttt{"\#\#\#"}$\\
    $S_4 = \texttt{"012345"}$\\
    执行的函数为:\\
    \textbf{concat(replace(S1, substr(S1, length(S2), length(S3)), S3), substr(S4, index(S2, '8'), length(S2)))}\\

    \textbf{第一步:计算 replace(S1, substr(S1, length(S2), length(S3)), S3)}\\
    计算 \textbf{length(S2)} 和 \textbf{length(S3)}:\\

    $S_2 = \texttt{"9898"}$,长度为 \textbf{4}。\\
    $S_3 = \texttt{"\#\#\#"}$,长度为 \textbf{3}。\\
    计算 \textbf{substr(S1, length(S2), length(S3))}:\\

    $S_1 = \texttt{"ABCDEFG"}$。\\
    \textbf{substr(S1, length(S2), length(S3))} 表示从 $S_1$ 的第 \textbf{4} 个字符开始(即 \texttt{D}),截取长度为 \textbf{3} 的子串。\\
    结果为:\texttt{"DEF"}。\\
    执行 \textbf{replace(S1, substr(S1, length(S2), length(S3)), S3)}:\\

    用 $S_3 = \texttt{"\#\#\#"}$ 替换 $S_1$ 中的子串 \texttt{"DEF"}。\\
    替换后的结果为:\texttt{"ABC\#\#\#G"}。\\

    \textbf{第二步:计算 substr(S4, index(S2, '8'), length(S2))}\\
    计算 \textbf{index(S2, '8')}:

    $S_2 = \texttt{"9898"}$。\\
    字符 \texttt{8} 第一次出现的位置是第 \textbf{2} 个字符(从 1 开始计数)。\\
    计算 \textbf{substr(S4, index(S2, '8'), length(S2))}:\\

    $S_4 = \texttt{"012345"}$。\\
    从 $S_4$ 的第 \textbf{2} 个字符开始(即 \texttt{1}),截取长度为 \textbf{length(S2) = 4} 的子串。\\
    结果为:\texttt{"2345"}。\\

    \textbf{第三步:执行 concat(...)}\\
    连接结果:\\
    前半部分:\texttt{"ABC\#\#\#G"}。\\
    后半部分:\texttt{"2345"}。\\
    连接后的结果为:\texttt{"ABC\#\#\#{G2345}"}。\\

    \textbf{最终答案:}\\
    C. \texttt{"ABC\#\#\#{G2345}"}\\
    \item 设有两个串 $S_1$ 和 $S_2$,求 $S_2$ 在 $S_1$ 中首次出现的位置的运算称作( )。  
    【中南大学 2005 一、3(2 分)】  

    A. 求子串 \quad B. 判断是否相等 \quad C. 模式匹配 \quad D. 连接  

    答案:C. 模式匹配
    解析:\\
    在计算机科学中,模式匹配是指在一个主串中查找一个模式串的过程。\\
    该过程通常涉及到比较主串和模式串的字符,以确定模式串在主串中首次出现的位置。\\
    因此,答案是 \textbf{\textcolor{red}{C}}. 模式匹配。\\

    \item 已知串 $S = \texttt{"aaab"}$,其 Next 数组值为( )。  
    【西安电子科技大学 1996 一、7(2 分)】 

    A. $(0, 1, 2, 3)$ \quad B. $(1, 1, 2, 3)$ \quad C. $(1, 2, 3, 1)$ \quad D. $(1, 2, 1, 1)$  

    答案:\textcolor{red}{A. $(0, 1, 2, 3)$}
    解析:\\
    在 KMP 算法中,Next 数组用于存储模式串中每个字符的最长相等前后缀的长度。\\
    对于串 \texttt{"ababaaababaa"},我们可以计算其 Next 数组:\\
    我们的讲义上面提过如何求解 Next 数组,我们先严格按照初值从-1 0 开始的next 数组来求解,\\
    1. 对于 下标 0 所在字符 \texttt{a},没有前缀,因此 Next[0] = -1(默认)。\\
    2. 对于 下标 1 所在字符 \texttt{a},没有前缀和后缀,因此 Next[1] = 0。\\
    3. 对于 下标 2 所在字符 \texttt{a},前缀为 \texttt{a},后缀为 \texttt{a},最长公共前后缀为 1,因此 Next[2] = 1。\\
    4. 对于 下标 3 所在字符 \texttt{b},前缀为 \texttt{a,aa},后缀为 \texttt{a,aa},最长公共前后缀为 2,因此 Next[3] = 2。\\

    综上,next 数组为 $(-1,0,1,2)$。\\
    但是我们在这里需要注意的是,Next 数组的定义有两种,一种是从 0 开始的 Next 数组,另一种是从 -1 开始的 Next 数组。\\
    在本题中,Next 数组是从 0 开始的,因此我们需要将 Next 数组的值加 1。\\
    因此,最终的 Next 数组为 $(0, 1, 2, 3)$。\\

    \item 串 \texttt{"ababaaababaa"} 的 Next 数组为( )。  
    【中山大学 1999 一、7;江苏大学 2006 一、1(2 分)】  

    A. $(0, 1, 2, 3, 4, 5, 6, 7, 8,9,9,9)$  

    B. $(0, 1, 2, 1, 2, 1, 1, 1, 1,2,1,2)$  

    C. $(0, 1, 1, 2, 3, 4, 2, 2, 3,4,5,6)$  

    D. $(0, 1, 2, 3, 0, 1, 2, 3, 2,2,3,4)$  

    答案:\textcolor{red}{C. $(0, 1, 1, 2, 3, 4, 2, 2, 3,4,5,6)$}\\
    解析:\\
    在 KMP 算法中,Next 数组用于存储模式串中每个字符的最长相等前后缀的长度。\\
    对于串 \texttt{"ababaaababaa"},我们可以计算其 Next 数组:\\
    我们的讲义上面提过如何求解 Next 数组,我们先严格按照初值从-1 0 开始的next 数组来求解,\\
    1. 对于 下标 0 所在字符 \texttt{a},没有前缀,因此 Next[0] = -1(默认)。\\
    2. 对于 下标 1 所在字符 \texttt{b},没有前缀和后缀,因此 Next[1] = 0。\\
    3. 对于 下标 2 所在字符 \texttt{a},前缀为 \texttt{a},后缀为 \texttt{b},无相等前后缀,因此 Next[2] = 0。\\
    4. 对于 下标 3 所在字符 \texttt{b},前缀为 \texttt{a,ab,},后缀为 \texttt{a,ba},最大公共前后缀长度为 1,因此 Next[3] = 1。\\
    5. 对于 下标 4 所在字符 \texttt{a},前缀为 \texttt{a,ab,aba,},后缀为 \texttt{b,ab,bab},最大公共前后缀长度为 2,因此 Next[4] = 2。\\
    6. 对于 下标 5 所在字符 \texttt{a},前缀为 \texttt{a,ab,aba,abab,},后缀为 \texttt{a,ba,aba,baba},最大公共前后缀长度为 3,因此 Next[5] = 3。\\
    7. 对于 下标 6 所在字符 \texttt{b},前缀为 \texttt{a,ab,aba,abab,ababa,},后缀为 \texttt{a,aa,baa,abaa,babaa},最大公共前后缀长度为 1,因此 Next[6] = 1。\\
    8. 对于 下标 7 所在字符 \texttt{a},前缀为 \texttt{a,ab,aba,abab,ababa,ababaa,},后缀为 \texttt{a,aa,aaa,baaa,abaaa,babaaa},最大公共前后缀长度为 1,因此 Next[7] = 1。\\
    9. 对于 下标 8 所在字符 \texttt{b},前缀为 \texttt{a,ab,aba,abab,ababa,ababaa,ababaaa,},后缀为 \texttt{b,ab,aab,aaab,baaab,abaaab,babaaab},最大公共前后缀长度为 2,因此 Next[8] = 2。\\
    10. 对于 下标 9 所在字符 \texttt{a},前缀为 \texttt{a,ab,aba,abab,ababa,ababaa,ababaaa,ababaaab,},后缀为 \texttt{a,ba,aba,aaba,aaaba,baaaba,abaaaba,babaaaba},最大公共前后缀长度为 3,因此 Next[9] = 3。\\
    11. 对于 下标 10 所在字符 \texttt{a},前缀为 \texttt{a,ab,aba,abab,ababa,ababaa,ababaaa,ababaaab,ababaaaba,},后缀为 \texttt{b,ab,bab,baab,aabaab,aaabaab,baaabaab,babaaabab},最大公共前后缀长度为 4,因此 Next[10] = 4。\\
    12. 对于 下标 11 所在字符 \texttt{b},前缀为 \texttt{a,ab,aba,abab,ababa,ababaa,ababaaa,ababaaab,ababaaaba,ababaaabaa,},后缀为 \texttt{a,aa,baa,baba,aababa,aaababa,baaababa,babaaababa},最大公共前后缀长度为 5,因此 Next[11] = 5。\\

    综上所述,Next 数组为 $(-1,0,0,1,2,3,1,1,2,3,4,5)$。\\
    但是我们在这里需要注意的是,Next 数组的定义有两种,一种是从 0 开始的 Next 数组,另一种是从 -1 开始的 Next 数组。\\
    在本题中,Next 数组是从 0 开始的,因此我们需要将 Next 数组的值加 1。\\
    因此,最终的 Next 数组为 $(0, 1, 1, 2, 3, 4, 2, 2, 3,4,5,6)$。\\
    \item 字符串 \texttt{"ababaabab"} 的 NextVal 数组为( )。  
    【北京邮电大学 1999 一、1(2 分);烟台大学 2007 一、8(2 分)】  

    A. $(0, 1, 0, 1, 0, 4, 1, 0, 1)$  

    B. $(0, 1, 0, 1, 0, 2, 1, 0, 1)$  

    C. $(0, 1, 0, 1, 0, 0, 0, 1, 1)$  

    D. $(0, 1, 0, 1, 0, 1, 0, 1, 1)$  

    答案:\textcolor{red}{A. $(0, 1, 0, 1, 0, 4, 1, 0, 1)$}\\

    解析:\\
    在 KMP 算法中,NextVal 数组是对 Next 数组的进一步优化,用于减少模式串的比较次数。\\
    对于串 \texttt{"ababaabab"},我们可以计算其 NextVal 数组:\\
    我们的讲义上面提过如何求解 Next 数组,我们先严格按照初值从-1 0 开始的next 数组来求解,\\
    1. 对于 下标 0 所在字符 \texttt{a},没有前缀,因此 Next[0] = -1(默认)。\\
    2. 对于 下标 1 所在字符 \texttt{b},没有前缀和后缀,因此 Next[1] = 0。\\
    3. 对于 下标 2 所在字符 \texttt{a},前缀为 \texttt{a},后缀为 \texttt{b},无相等前后缀,因此 Next[2] = 0。\\
    4. 对于 下标 3 所在字符 \texttt{b},前缀为 \texttt{a,ab,},后缀为 \texttt{a,ba},最大公共前后缀长度为 1,因此 Next[3] = 1。\\
    5. 对于 下标 4 所在字符 \texttt{a},前缀为 \texttt{a,ab,aba,},后缀为 \texttt{b,ab,bab},最大公共前后缀长度为 2,因此 Next[4] = 2。\\
    6. 对于 下标 5 所在字符 \texttt{a},前缀为 \texttt{a,ab,aba,abab,},后缀为 \texttt{a,ba,aba,baba},最大公共前后缀长度为 3,因此 Next[5] = 3。\\
    7. 对于 下标 6 所在字符 \texttt{b},前缀为 \texttt{a,ab,aba,abab,ababa,},后缀为 \texttt{a,aa,baa,abaa,babaa},最大公共前后缀长度为 1,因此 Next[6] = 1。\\
    8. 对于 下标 7 所在字符 \texttt{a},前缀为 \texttt{a,ab,aba,abab,ababa,ababaa,},后缀为 \texttt{b,ab,aab,baab,abaab,babaab},最大公共前后缀长度为 1,因此 Next[7] = 2。\\
    9. 对于 下标 8 所在字符 \texttt{b},前缀为 \texttt{a,ab,aba,abab,ababa,ababaa,ababaaa,},后缀为 \texttt{a,ba,aba,aaba,baaba,abaaba,babaaba},最大公共前后缀长度为 2,因此 Next[8] = 3。\\
    综上所述,Next 数组为 $(-1,0,0,1,2,3,1,2,3)$。\\

    我们可以通过next 数组来计算 NextVal 数组。\\
    
    1. 对于 下标 8 ,我们可以发现str[8] == str[next[8]] == str[next[next[8]]]=str[next[next[next[9]]]] ,因此我们可以将next[8] 赋值为 next[next[next[8]]] = 0。\\

    2. 对于 下标 7 ,我们可以发现str[7] == str[next[7]] == str[next[next[7]]] ,因此我们可以将next[7] 赋值为 next[next[7]] = -1。\\
    3. 对于 下标 6 ,我们可以发现str[6] == str[next[6]] == str[next[next[6]]] ,因此我们可以将next[6] 赋值为 next[next[6]] = 0。\\

    同理,可以求出其他下标的值。\\

    最终求得的nextVal值为 $(-1,0,-1,0,-1,3,0,-1,0)$。\\
    但是我们在这里需要注意的是,NextVal 数组的定义有两种,一种是从 0 开始的 NextVal 数组,另一种是从 -1 开始的 NextVal 数组。\\
    在本题中,NextVal 数组是从 0 开始的,因此我们需要将 NextVal 数组的值加 1。\\
    因此,最终的 NextVal 数组为 $(0,1,0,1,0,4,1,0,1)$。\\

    \item 模式串 $P = \texttt{"abcaabbcabcaabdab"}$,该模式串的 Next 数组的值为( ),NextVal 数组的值为( )。  
    
    A. 0 1 1 1 2 2 1 1 1 2 3 4 5 6 7 1 2 

    B. 0 1 1 1 2 1 2 1 1 2 3 4 5 6 1 1 2 

    C. 0 1 1 1 0 0 1 3 1 0 1 1 0 0 7 0 1 

    D. 0 1 1 1 2 2 3 1 1 2 3 4 5 6 7 1 2 

    E. 0 1 1 0 0 1 1 1 0 1 1 0 0 1 7 0 1 

    F. 0 1 1 0 2 1 3 1 0 1 0 1 2 1 7 0 1 

    答案:\textcolor{red}{A. 0 1 1 1 2 2 1 1 1 2 3 4 5 6 7 1 2}\\ 
    \textcolor{red}{D. 0 1 1 1 2 2 3 1 1 2 3 4 5 6 7 1 2}\\
    解析:\\
    在 KMP 算法中,Next 数组用于存储模式串中每个字符的最长相等前后缀的长度。\\
    对于模式串 $P = \texttt{"abcaabbcabcaabdab"}$,我们可以计算其 Next 数组:\\
    我们的讲义上面提过如何求解 Next 数组,我们先严格按照初值从-1 0 开始的next 数组来求解,\\
    1. 对于 下标 0 所在字符 \texttt{a},没有前缀,因此 Next[0] = -1(默认)。\\
    2. 对于 下标 1 所在字符 \texttt{b},没有前缀和后缀,因此 Next[1] = 0。\\
    3. 对于 下标 2 所在字符 \texttt{c},前缀为 \texttt{a},后缀为 \texttt{b},无相等前后缀,因此 Next[2] = 0。\\
    4. 对于 下标 3 所在字符 \texttt{a},前缀为 \texttt{a,ab,},后缀为 \texttt{c,bc},最大公共前后缀长度为 0,因此 Next[3] = 0。\\
    5. 对于 下标 4 所在字符 \texttt{a},前缀为 \texttt{a,ab,abc,},后缀为 \texttt{a,ca,bca},最大公共前后缀长度为 1,因此 Next[4] = 1。\\

    同理,求出next数组为 $(-1,0,0,0,1,1,0,0,0,1,2,3,4,5,6,0,1)$\\
    
    将next数组转换为nextval数组,简单来说就是观察next数组,如果next[i] == next[next[i]] ,那么就将next[i] 赋值为 next[next[i]]。\\
    为了清晰观察,我们将字符串和next数组对应起来,如下表所示:\\
    \begin{center}
    \begin{tabular}{|c|c|c|c|c|c|c|c|c|c|c|c|c|c|c|c|c|c|}
    \hline
    \textbf{下标} & 0 & 1 & 2 & 3 & 4 & 5 & 6 & 7 & 8 & 9 & 10 & 11 & 12 & 13 & 14 & 15 & 16 \\
    \hline
    \textbf{字符} & a & b & c & a & a & b & b & c & a & b & c & a & a & b & d & a & b \\
    \hline
    \textbf{next} & -1 & 0 & 0 & 0 & 1 & 1 & 0 & 0 & 0 & 1 & 2 & 3 & 4 & 5 & 6 & 0 & 1 \\
    \hline
    \end{tabular}
    \end{center}

    求解nextVal数组的过程如下:\\
    1. 对于下标16,我们可以发现str[16] == str[next[16]] !== str[next[next[16]]] ,因此我们可以将next[16] 赋值为 next[next[16]] = 0。\\
    2. 对于下标15,我们可以发现str[15] == str[next[15]] ,因此我们可以将next[15] 赋值为 next[next[15]] = -1。\\
    \dots\dots\\

    综上,我们可以得到nextVal数组为 $(-1,0,0,0,1,1,0,0,0,1,2,3,4,5,6,-1,0)$。\\
    但是我们在这里需要注意的是,NextVal 数组的定义有两种,一种是从 0 开始的 NextVal 数组,另一种是从 -1 开始的 NextVal 数组。\\
    在本题中,NextVal 数组是从 0 开始的,因此我们需要将 NextVal 数组的值加 1。\\
    因此,最终的 NextVal 数组为 $(0,1,1,1,2,2,3,1,1,2,3,4,5,6,7,1,2)$。\\

    \item 若串 $S = \texttt{"myself"}$,则其子串的数目是( )。  
    【北京理工大学 2007 一、6(1 分)】  

    A. 20 \quad B. 21 \quad C. 22 \quad D. 23  

    
    答案:\textcolor{red}{B. 21}\\
    解析:\\
    对于一个长度为 $n$ 的字符串,其子串的个数可以通过以下公式计算:\\
    \[
    \text{子串个数} = \frac{n(n+1)}{2}
    \]
    其中 $n$ 是字符串的长度。\\
    在本题中,字符串 $S = \texttt{"myself"}$ 的长度为 6。\\
    因此,子串的个数为:\\
    \[
        \text{子串个数} = \frac{6(6+1)}{2} = \frac{6 \times 7}{2} = 21
    \]

    所以,答案是 \textbf{\textcolor{red}{B}}. 21。\\

    \item 若串 $S = \texttt{"software"}$,则其子串的数目是( )。  
    【西安电子科技大学 2001 应用一、2(2 分)】  

    A. 8 \quad B. 37 \quad C. 36 \quad D. 9  

    答案:\textcolor{red}{C. 36}\\
    解析:\\
    对于一个长度为 $n$ 的字符串,其子串的个数可以通过以下公式计算:\\
    \[
    \text{子串个数} = \frac{n(n+1)}{2}
    \]
    其中 $n$ 是字符串的长度。\\
    在本题中,字符串 $S = \texttt{"software"}$ 的长度为 8。\\
    因此,子串的个数为:\\
    \[
        \text{子串个数} = \frac{8(8+1)}{2} = \frac{8 \times 9}{2} = 36
    \]
    所以,答案是 \textbf{\textcolor{red}{C}}. 36。\\

    \item 设 $S$ 是一个长度为 $m$ 的字符串,其中的字符各不相同,则 $S$ 中互异的非平凡子串(非空且不同于 $S$ 本身)的个数为( )。  
    【中科院计算所 1997;烟台大学 2007 一、7(2 分)】  

    A. $2m - 1$  

    B. $m^2$  

    C. $\frac{m^2}{2} + m/2$  

    D. $\frac{m^2}{2} + m/2 - 1$  

    E. $\frac{m^2}{2} - m/2 - 1$  

    F. 其他情况  

    答案:\textcolor{red}{C. $\frac{m^2}{2} + m/2$}\\
    解析:\\
    对于一个长度为 $m$ 的字符串,其中的字符各不相同,则 $S$ 中互异的非平凡子串(非空且不同于 $S$ 本身)的个数可以通过以下公式计算:\\
    \[
    \text{子串个数} = \frac{m(m+1)}{2} - 1
    \]
    其中 $m$ 是字符串的长度。\\
    在本题中,字符串 $S$ 的长度为 $m$。\\
    因此,子串的个数为:\\
    \[
        \text{子串个数} = \frac{m(m+1)}{2} - 1
    \]
    但是我们需要注意的是,题目中要求的是互异的非平凡子串,因此我们需要将结果减去 1。\\
    所以,答案是 \textbf{\textcolor{red}{C}}. $\frac{m^2}{2} + m/2$。\\

    \item 串是一种特殊的线性表,其特殊性体现在( )。  
    【暨南大学 2010 一、11(2 分)】  

    A. 可以顺序存储  

    B. 数据元素是一个字符  

    C. 可以链接存储  

    D. 数据元素可以是多个字符  

    答案:\textcolor{red}{D. 数据元素可以是多个字符}\\
    解析:\\
    串是一种特殊的线性表,其特殊性体现在数据元素可以是多个字符。\\
    串的定义是由零个或多个字符组成的有限序列。\\
    因此,答案是 \textbf{\textcolor{red}{D}}. 数据元素可以是多个字符。\\

    \item 在下列表述中,( )是错误的。  
    【华中科技大学 2006 二、2(2 分)】  


    A. 含有一个或多个空格字符的串称为空格串  

    B. 对 $n$ ($n > 0$) 个顶点的网,求出权最小的 $n-1$ 条边便可构成其最小生成树  

    C. 选择排序算法是不稳定的  

    D. 平衡二叉树的左右子树的结点数之差的绝对值不超过 1   

    答案:\textcolor{red}{A. 含有一个或多个空格字符的串称为空格串}\\
    解析:\\
    在计算机科学中,空格串是指只包含空格字符的字符串。\\
    含有一个或多个空格字符的串称为空格串是错误的。\\
    因此,答案是 \textbf{\textcolor{red}{A}}. 含有一个或多个空格字符的串称为空格串。\\
\end{enumerate}
\end{document}
